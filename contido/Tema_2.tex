%%%%%%%%%%%%%%%%%%%%%%%%%%%%%%%%%%%%%%%%%%%%%%%%%%%%%%%%%%%%%%%%%%%%%%%%
% Plantilla TFG/TFM
% Universidad de A Coruña. Facultad de Informática
% Realizado por: Welton Vieira dos Santos
% Modificado: Welton Vieira dos Santos
% Contacto: welton.dossantos@udc.es
%%%%%%%%%%%%%%%%%%%%%%%%%%%%%%%%%%%%%%%%%%%%%%%%%%%%%%%%%%%%%%%%%%%%%%%%


\chapter{Análisis de Impactos y Mejoras: Modelado de las Tareas y los Agentes.}
\newpage
\section{Formulario TM-1: análisis de tareas.}
Descripción detallada de tareas en el contexto del proceso de interés.

% Formulario TM-1 de la tarea 1
\begin{table}[H]
	\scriptsize
  	\resizebox{14,2cm}{!}{
		\begin{tabularx}{\textwidth}{|l|X|} 
			\hline
			\textbf{Modelo de Tareas} & \textbf{Formulario TM-1: Análisis de Tareas} \\ 
			\hline
			\hline

			\textsc{Tarea} & \textbf{Tarea 1:} \textit{Recibimiento de los paquetes en la sucursal MRW (entrada)} \\ 
			\hline
			\textsc{Organización}  & \textit{Logística.} \\ 
			\hline
			\textsc{Objetivo y valor} &  \textit{Es una parte obligatoria de todo el proceso.} \\ 
			\hline
			\textsc{Dependencia y Flujos} & 
				\begin{enumerate}
					\item \textbf{Tareas precedentes:} \textit{Ninguna}
					\item \textbf{Tareas que le siguen:} \textit{Tarea 2}
				\end{enumerate} \\
			\hline
			\textsc{Objetos manipulados} & 
				\begin{enumerate}
					\item \textbf{Objetos de entrada de la tarea:} \textit{Información de los datos de los paquetes (a través de las herramientas de lectura de códigos de barras o códigos QR).}
					\item \textbf{Objetos de salida de la tarea:} \textit{Confirmación de recibimiento del paquete.}
					\item \textbf{Objetos internos:} \textit{Almacena la información del paquete de entrada en una base de datos de paquetería de la sucursal.} 
				\end{enumerate} \\ 
			\hline
			\textsc{Tiempo y control} & 
				\begin{enumerate}
					\item \textbf{Frecuencia y duración:} \textit{Es una tarea que se realiza toda las veces que se da entrada de nueva paquetería y la duración es instantánea}
					\item \textbf{Control:} \textit{respecto a otras  tareas, ninguna.}
					\item \textbf{Restricciones:} \textit{Se necesita una conexión permanente con un sistema que suministra MRW para la consulta de la información de cada paquete.}
				\end{enumerate} \\
			\hline
			\textsc{Agentes} & 
				\begin{enumerate}
					\item \textbf{Repartidor:} \textit{Descargue los paquetes del transporte (furgoneta o camión).}
					\item \textbf{Lector de etiquetas:} \textit{Utilizado para identificar la entrada de paquetería en la sucursal.}
				\end{enumerate} \\
			\hline
			\textsc{Conocimiento y Capacidad} & \textit{No necesita ningún conocimiento específico.} \\
			\hline
			\textsc{Recursos} & 
				\begin{itemize}
					\item \textit{Identificador de etiquetas.}
					\item \textit{Mano de obra para efectuar la descarga de la paquetería.}
					\item \textit{Sistema de gestión de control de MRW (base de datos de la paquetería).}
				\end{itemize} \\
			\hline
			\textsc{Calidad y eficiencia} & \textit{Seguir unas pautas para descargar la paquetería, es decir, hay determinados paquetes especiales que tiene un trato diferenciado.} \\
			\hline
		\end{tabularx}
	}
	\caption{\label{tab:TM1T1}Formulario TM-1: Analisis de tarea 1 del OM-3}
\end{table} 

% Formulario TM-1 de la tarea 2
\begin{table}[H]
	\scriptsize
  	\resizebox{14,2cm}{!}{
		\begin{tabularx}{\textwidth}{|l|X|} 
			\hline
			\textbf{Modelo de Tareas} & \textbf{Formulario TM-1: Análisis de Tareas} \\ 
			\hline
			\hline

			\textsc{Tarea} & \textbf{Tarea 2:} \textit{Generar lista de entregas según ruta asignada.} \\ 
			\hline
			\textsc{Organización}  & \textit{Logística.} \\ 
			\hline
			\textsc{Objetivo y valor} &  \textit{Es una parte obligatoria de todo el proceso.} \\ 
			\hline
			\textsc{Dependencia y Flujos} & 
				\begin{enumerate}
					\item \textbf{Tareas precedentes:} \textit{Tarea 1}
					\item \textbf{Tareas que le siguen:} \textit{Tarea 2}
				\end{enumerate} \\
			\hline
			\textsc{Objetos manipulados} & 
				\begin{enumerate}
					\item \textbf{Objetos de entrada de la tarea:} \textit{Información en la etiqueta del paquete.}
					\item \textbf{Objetos de salida de la tarea:} \textit{Información de la ruta que pertenece el paquete.}
					\item \textbf{Objetos internos:} \textit{Hoja de Excel preparada para dejar constancia de los paquetes de cada ruta.} 
				\end{enumerate} 
				\emph{El objeto de salida incluye elementos de conocimiento por parte del repartidor experimentado y en caso de duda, se consulta el equipo directivo.}\\				
			\hline
			\textsc{Tiempo y control} & 
				\begin{enumerate}
					\item \textbf{Frecuencia y duración:} \textit{Es una tarea que se realiza a la primera hora de cada jornada. La duración es de una hora y media (aproximadamente).}
					\item \textbf{Control:} \textit{Cada paquete tiene que haber sido procesado en la tarea 1 obrigatoriamente.}
					\item \textbf{Restricciones:} \textit{Ninguna restricción, exepto la del control.}
				\end{enumerate} \\
			\hline
			\textsc{Agentes} & 
				\begin{enumerate}
					\item \textbf{Repartidor/Equipo directivo:} \textit{Va apuntando en una hoja de excel los paquetes que pertenece a su ruta.}
				\end{enumerate} \\
			\hline
			\textsc{Conocimiento y Capacidad} & \textit{Experiencia por parte del repartidor o en su caso, el equipo directivo.} \\
			\hline
			\textsc{Recursos} & 
				\begin{itemize}
					\item \textit{Ordenador con una hora de Excel preparada para almacenar los paquetes de cada ruta.}
					\item \textit{Acceso a la base de datos de paquetería de MRW.}
					\item \textit{Impresora para imprimir la lista de cada ruta.}
				\end{itemize} \\
			\hline
			\textsc{Calidad y eficiencia} & \textit{Seguir unas pautas de procesamiento de cada paquete, es decir, no olvidar de inscribir cada paquete en la hoja de excel de la ruta asignada.} \\
			\hline
		\end{tabularx}
	}
	\caption{\label{tab:TM1T2}Formulario TM-1: Analisis de tarea 2 del OM-3}
\end{table} 

% Formulario TM-1 de la tarea 3
\begin{table}[H]
	\scriptsize
  	\resizebox{14,2cm}{!}{
		\begin{tabularx}{\textwidth}{|l|X|} 
			\hline
			\textbf{Modelo de Tareas} & \textbf{Formulario TM-1: Análisis de Tareas} \\ 
			\hline
			\hline

			\textsc{Tarea} & \textbf{Tarea 3:} \textit{Determinar los recursos disponibles por la sucursal MRW para entregar según ruta.} \\ 
			\hline
			\textsc{Organización}  & \textit{Logística.} \\ 
			\hline
			\textsc{Objetivo y valor} &  \textit{Es una parte obligatoria de todo el proceso.} \\ 
			\hline
			\textsc{Dependencia y Flujos} & 
				\begin{enumerate}
					\item \textbf{Tareas precedentes:} \textit{Ninguna}
					\item \textbf{Tareas que le siguen:} \textit{Tarea 2}
				\end{enumerate} \\
			\hline
			\textsc{Objetos manipulados} & 
				\begin{enumerate}
					\item \textbf{Objetos de entrada de la tarea:} \textit{Información de los datos de los paquetes (a través de las herramientas de lectura de códigos de barras o códigos QR), Información recibida por el sistema de gestión de paquetes de MRW (Sistema suministrado por la empresa autora de la franquicia).}
					\item \textbf{Objetos de salida de la tarea:} \textit{información de la ruta adjudicada al paquete.}
					\item \textbf{Objetos internos:} \textit{conocimiento de los expertos en adjudicar a que ruta pertenece el paquete.} 
				\end{enumerate} 
				\emph{Todos estos objetos incluyen elementos de información y conocimiento.}\\				
			\hline
			\textsc{Tiempo y control} & 
				\begin{enumerate}
					\item \textbf{Frecuencia y duración:} \textit{es una tarea que se da cuando se considera  oportuna (en función de la información y conocimiento del sistema).}
					\item \textbf{Control:} \textit{respecto a otras  tareas, ninguna.}
					\item \textbf{Restricciones:} \textit{Se necesita una conexión permanente con un sistema que suministra MRW para la consulta de la información de cada paquete.}
				\end{enumerate} \\
			\hline
			\textsc{Agentes} & 
				\begin{enumerate}
					\item \textbf{Repartidor:} \textit{Coloca el paquete en la cinta transportadora}
					\item \textbf{Sistema de gestión de paquetes de MRW:} \textit{Permite hacer una consulta en busca de la información del paquete.}
				\end{enumerate} \\
			\hline
			\textsc{Conocimiento y Capacidad} & \textit{No exige ninguna experiencia por parte del repartidor, ya que el mismo sólo tiene que poner el paquete en la cinta transportadora.} \\
			\hline
			\textsc{Recursos} & 
				\begin{itemize}
					\item \textit{Cinta transportadora equipada con sensor de lectura de las etiquetas que se encuentra en el paquete.}
					\item \textit{Acceso a la base de datos de paquetería de MRW.}
				\end{itemize} \\
			\hline
			\textsc{Calidad y eficiencia} & \textit{Seguir unas pautas de colocación de los paquetes en la cinta transportadora para que el lector de etiquetas funcione correctamente y evite la paralización de la misma.} \\
			\hline
		\end{tabularx}
	}
	\caption{\label{tab:TM1T3}Formulario TM-1: Analisis de tarea 3 del OM-3}
\end{table} 

% Formulario TM-1 de la tarea 4
\begin{table}[H]
	\scriptsize
  	\resizebox{14,2cm}{!}{
		\begin{tabularx}{\textwidth}{|l|X|} 
			\hline
			\textbf{Modelo de Tareas} & \textbf{Formulario TM-1: Análisis de Tareas} \\ 
			\hline
			\hline

			\textsc{Tarea} & \textbf{Tarea 4:} \textit{Revisión y validación de la distribución de la paquetería.} \\ 
			\hline
			\textsc{Organización}  & \textit{Logística.} \\ 
			\hline
			\textsc{Objetivo y valor} &  \textit{Es una parte obligatoria de todo el proceso.} \\ 
			\hline
			\textsc{Dependencia y Flujos} & 
				\begin{enumerate}
					\item \textbf{Tareas precedentes:} \textit{Ninguna}
					\item \textbf{Tareas que le siguen:} \textit{Tarea 2}
				\end{enumerate} \\
			\hline
			\textsc{Objetos manipulados} & 
				\begin{enumerate}
					\item \textbf{Objetos de entrada de la tarea:} \textit{Información de los datos de los paquetes (a través de las herramientas de lectura de códigos de barras o códigos QR), Información recibida por el sistema de gestión de paquetes de MRW (Sistema suministrado por la empresa autora de la franquicia).}
					\item \textbf{Objetos de salida de la tarea:} \textit{información de la ruta adjudicada al paquete.}
					\item \textbf{Objetos internos:} \textit{conocimiento de los expertos en adjudicar a que ruta pertenece el paquete.} 
				\end{enumerate} 
				\emph{Todos estos objetos incluyen elementos de información y conocimiento.}\\				
			\hline
			\textsc{Tiempo y control} & 
				\begin{enumerate}
					\item \textbf{Frecuencia y duración:} \textit{es una tarea que se da cuando se considera  oportuna (en función de la información y conocimiento del sistema).}
					\item \textbf{Control:} \textit{respecto a otras  tareas, ninguna.}
					\item \textbf{Restricciones:} \textit{Se necesita una conexión permanente con un sistema que suministra MRW para la consulta de la información de cada paquete.}
				\end{enumerate} \\
			\hline
			\textsc{Agentes} & 
				\begin{enumerate}
					\item \textbf{Repartidor:} \textit{Coloca el paquete en la cinta transportadora}
					\item \textbf{Sistema de gestión de paquetes de MRW:} \textit{Permite hacer una consulta en busca de la información del paquete.}
				\end{enumerate} \\
			\hline
			\textsc{Conocimiento y Capacidad} & \textit{No exige ninguna experiencia por parte del repartidor, ya que el mismo sólo tiene que poner el paquete en la cinta transportadora.} \\
			\hline
			\textsc{Recursos} & 
				\begin{itemize}
					\item \textit{Cinta transportadora equipada con sensor de lectura de las etiquetas que se encuentra en el paquete.}
					\item \textit{Acceso a la base de datos de paquetería de MRW.}
				\end{itemize} \\
			\hline
			\textsc{Calidad y eficiencia} & \textit{Seguir unas pautas de colocación de los paquetes en la cinta transportadora para que el lector de etiquetas funcione correctamente y evite la paralización de la misma.} \\
			\hline
		\end{tabularx}
	}
	\caption{\label{tab:TM1T4}Formulario TM-1: Analisis de tarea 4 del OM-3}
\end{table} 

\newpage

\section{Formulario TM-2: análisis de los cuellos de botella del conocimiento.}
Especificación del conocimiento que se emplea en una tarea, sus cuellos de botella y posibles mejoras.

\begin{table}[H]
	\centering
	\resizebox{15.0cm}{!}{
	  \begin{tabular}{|l|l|l|} 
		\hline
		\textbf{Modelo de Tareas} & \multicolumn{2}{p{15.0cm}|}{\textbf{Formulario TM-2: Elementos de conocimiento}}\\ 
		\hline\hline

		\textsc{Nombre} & \multicolumn{2}{p{15.0cm}|}{Experiencia en gestionar las operativas de compra y venta de activos al mercado de divisas} \\
		\hline

		\textsc{Poseído por} & \multicolumn{2}{p{15.0cm}|}{Expertos en especulación en mercados de activos} \\
		\hline

		\textsc{Usado en} & \multicolumn{2}{p{15.0cm}|}{Tarea 5 - Gestionar las operativas abiertas} \\
		\hline

		\textsc{Dominio} & \multicolumn{2}{p{15.0cm}|}{Inversión y especulación en bolsa, ámbito económico} \\
		\hline

		\textsc{\textbf{Naturaleza del conocimiento}} & \multicolumn{1}{p{1.2cm}|}{\centering \textit{\textbf{Si/No}}} & \multicolumn{1}{p{13.0cm}|}{\textbf{¿Cuello de botella/debe ser mejorado?}}\\
		\hline

		Formal, riguroso & \multicolumn{1}{p{1.0cm}|}{No} & \multicolumn{1}{p{13.0cm}|}{No}\\
		\hline

		Empírico, cuantitativo & \multicolumn{1}{p{1.0cm}|}{No} & \multicolumn{1}{p{13.0cm}|}{No}\\
		\hline

		Heurístico, sentido común & \multicolumn{1}{p{1.0cm}|}{Si} & \multicolumn{1}{p{13.0cm}|}{Si, no es fácil de transferir, si es mejorable}\\
		\hline

		\multicolumn{1}{|p{6.0cm}|}{Altamente especializado, específico del dominio} & \multicolumn{1}{p{1.0cm}|}{Si} & \multicolumn{1}{p{13.0cm}|}{Si, se necesita conocimientos del dominio, es mejorable}\\
		\hline

		Basado en la experiencia & \multicolumn{1}{p{1.0cm}|}{Si} & \multicolumn{1}{p{13.0cm}|}{Si, hay una dependencia enorme del experto,es mejorable}\\
		\hline

		Basado en la acción & \multicolumn{1}{p{1.0cm}|}{No} & \multicolumn{1}{p{13.0cm}|}{No}\\
		\hline

		Incompleto & \multicolumn{1}{p{1.0cm}|}{No} & \multicolumn{1}{p{13.0cm}|}{No}\\
		\hline

		Incierto, puede ser incorrecto & \multicolumn{1}{p{1.0cm}|}{Si} & \multicolumn{1}{p{13.0cm}|}{Si, pero es impredicible, no es mejorable}\\
		\hline

		Cambia con rapidez & \multicolumn{1}{p{1.0cm}|}{No} & \multicolumn{1}{p{13.0cm}|}{No, pero es cierto que se hay que adaptar al mercado.}\\
		\hline

		Dificil de verificar & \multicolumn{1}{p{1.0cm}|}{No} & \multicolumn{1}{p{13.0cm}|}{No}\\
		\hline

		Tácito, dificil de transferir& \multicolumn{1}{p{1.0cm}|}{Si} & \multicolumn{1}{p{13.0cm}|}{Si, hay que encontrar una forma de plasmarlo}\\
		\hline

		\textsc {\textbf{Forma del conocimiento}}& \multicolumn{1}{p{1.0cm}|}{} & \multicolumn{1}{p{13.0cm}|}{}\\
		\hline

		Mental & \multicolumn{1}{p{1.0cm}|}{Si} & \multicolumn{1}{p{13.0cm}|}{Si, debemos pasar a un medio que nos permita trabajar con él}\\
		\hline

		Papel & \multicolumn{1}{p{1.0cm}|}{No} & \multicolumn{1}{p{13.0cm}|}{No}\\
		\hline

		Electrónica & \multicolumn{1}{p{1.0cm}|}{No} & \multicolumn{1}{p{13.0cm}|}{No}\\
		\hline

		Habilidades & \multicolumn{1}{p{1.0cm}|}{Si} & \multicolumn{1}{p{13.0cm}|}{Si, es mejorable}\\
		\hline

		Otros & \multicolumn{1}{p{1.0cm}|}{No} & \multicolumn{1}{p{13.0cm}|}{}\\
		\hline

		\textsc {\textbf{Disponibilidad del Conocimiento}} & \multicolumn{1}{p{1.0cm}|}{} & \multicolumn{1}{p{13.0cm}|}{}\\
		\hline
		Limitaciones de tiempo& \multicolumn{1}{p{1.0cm}|}{Si} & \multicolumn{1}{p{13.0cm}|}{Si, dependemos de los expertos}\\
		\hline

		Limitaciones de espacio& \multicolumn{1}{p{1.0cm}|}{No} & \multicolumn{1}{p{13.0cm}|}{}\\
		\hline

		Limitaciones de acceso& \multicolumn{1}{p{1.0cm}|}{Si} & \multicolumn{1}{p{13.0cm}|}{Si, dependemos de los expertos}\\
		\hline

		Limitaciones de calidad& \multicolumn{1}{p{1.0cm}|}{Si} & \multicolumn{1}{p{13.0cm}|}{Si, depende de la calidad de los expertos}\\
		\hline

		Limitaciones de forma& \multicolumn{1}{p{1.0cm}|}{No} & \multicolumn{1}{p{13.0cm}|}{}\\
		\hline

	  \end{tabular}
	}
	\caption{\label{tab:TM2}Formulario TM-2: Analisis de cuellos de botella en la Tarea 5 - (Experiencia en gestionar...)}
  \end{table}

  \begin{table}[H]
	\centering
	\resizebox{15.0cm}{!}{
	  \begin{tabular}{|l|l|l|} 
		\hline
		\textbf{Modelo de Tareas} & \multicolumn{2}{p{15.0cm}|}{\textbf{Formulario TM-2: Elementos de conocimiento}}\\ 
		\hline\hline

		\textsc{Nombre} & \multicolumn{2}{p{15.0cm}|}{Teorías de gestión de capital de Inversión} \\
		\hline

		\textsc{Poseído por} & \multicolumn{2}{p{15.0cm}|}{Expertos en especulación en mercados de activos} \\
		\hline

		\textsc{Usado en} & \multicolumn{2}{p{15.0cm}|}{Tarea 5 - Gestionar las operativas abiertas} \\
		\hline

		\textsc{Dominio} & \multicolumn{2}{p{15.0cm}|}{Inversión y especulación en bolsa, ámbito económico} \\
		\hline

		\textsc{\textbf{Naturaleza del conocimiento}} & \multicolumn{1}{p{1.2cm}|}{\centering \textit{\textbf{Si/No}}} & \multicolumn{1}{p{13.0cm}|}{\textbf{¿Cuello de botella/debe ser mejorado?}}\\
		\hline

		Formal, riguroso & \multicolumn{1}{p{1.0cm}|}{Si} & \multicolumn{1}{p{13.0cm}|}{No}\\
		\hline

		Empírico, cuantitativo & \multicolumn{1}{p{1.0cm}|}{Si} & \multicolumn{1}{p{13.0cm}|}{No}\\
		\hline

		Heurístico, sentido común & \multicolumn{1}{p{1.0cm}|}{No} & \multicolumn{1}{p{13.0cm}|}{No}\\
		\hline

		\multicolumn{1}{|p{6.0cm}|}{Altamente especializado, específico del dominio} & \multicolumn{1}{p{1.0cm}|}{Si} & \multicolumn{1}{p{13.0cm}|}{Si, se necesita conocimientos del dominio, es mejorable}\\
		\hline

		Basado en la experiencia & \multicolumn{1}{p{1.0cm}|}{No} & \multicolumn{1}{p{13.0cm}|}{No}\\
		\hline

		Basado en la acción & \multicolumn{1}{p{1.0cm}|}{No} & \multicolumn{1}{p{13.0cm}|}{No}\\
		\hline

		Incompleto & \multicolumn{1}{p{1.0cm}|}{No} & \multicolumn{1}{p{13.0cm}|}{No}\\
		\hline

		Incierto, puede ser incorrecto & \multicolumn{1}{p{1.0cm}|}{No} & \multicolumn{1}{p{13.0cm}|}{No}\\
		\hline

		Cambia con rapidez & \multicolumn{1}{p{1.0cm}|}{No} & \multicolumn{1}{p{13.0cm}|}{No}\\
		\hline

		Dificil de verificar & \multicolumn{1}{p{1.0cm}|}{No} & \multicolumn{1}{p{13.0cm}|}{No}\\
		\hline

		Tácito, dificil de transferir& \multicolumn{1}{p{1.0cm}|}{No} & \multicolumn{1}{p{13.0cm}|}{No}\\
		\hline

		\textsc {\textbf{Forma del conocimiento}}& \multicolumn{1}{p{1.0cm}|}{} & \multicolumn{1}{p{13.0cm}|}{}\\
		\hline

		Mental & \multicolumn{1}{p{1.0cm}|}{No} & \multicolumn{1}{p{13.0cm}|}{No}\\
		\hline

		Papel & \multicolumn{1}{p{1.0cm}|}{Si} & \multicolumn{1}{p{13.0cm}|}{No}\\
		\hline

		Electrónica & \multicolumn{1}{p{1.0cm}|}{No} & \multicolumn{1}{p{13.0cm}|}{No}\\
		\hline

		Habilidades & \multicolumn{1}{p{1.0cm}|}{No} & \multicolumn{1}{p{13.0cm}|}{No}\\
		\hline

		Otros & \multicolumn{1}{p{1.0cm}|}{No} & \multicolumn{1}{p{13.0cm}|}{No}\\
		\hline

		\textsc {\textbf{Disponibilidad del Conocimiento}} & \multicolumn{1}{p{1.0cm}|}{} & \multicolumn{1}{p{13.0cm}|}{}\\
		\hline
		Limitaciones de tiempo& \multicolumn{1}{p{1.0cm}|}{No} & \multicolumn{1}{p{13.0cm}|}{No}\\
		\hline

		Limitaciones de espacio& \multicolumn{1}{p{1.0cm}|}{No} & \multicolumn{1}{p{13.0cm}|}{No}\\
		\hline

		Limitaciones de acceso& \multicolumn{1}{p{1.0cm}|}{No} & \multicolumn{1}{p{13.0cm}|}{No}\\
		\hline

		Limitaciones de calidad& \multicolumn{1}{p{1.0cm}|}{No} & \multicolumn{1}{p{13.0cm}|}{No}\\
		\hline

		Limitaciones de forma& \multicolumn{1}{p{1.0cm}|}{No} & \multicolumn{1}{p{13.0cm}|}{No}\\
		\hline

	  \end{tabular}
	}
	\caption{\label{tab:TM22}Formulario TM-2: Analisis de cuellos de botella en la Tarea 5 - (Teorías de gestión de capital...)}
  \end{table}

\newpage
\section{Formulario AM-1: descripción de los agentes.}
Descripción de los agentes implicados en las tareas de interés.

\begin{table}[H]
	\centering
	\resizebox{15.0cm}{!}{
	  \begin{tabular}{|l|l|} 
		\hline
		\textbf{Modelo de Agentes} & \textbf{Formulario TM-1: Agente}\\ 
		\hline\hline
		\textsc{Nombre} & \multicolumn{1}{p{15.0cm}|}{Experto en especulación mercado activos} \\
		\hline

		\textsc{Organización} & \multicolumn{1}{p{15.0cm}|}{Departamento de Análisis}\\
		\hline

		\textsc{Implicado en} & \multicolumn{1}{p{15.0cm}|}{Todas las tareas}\\
		\hline

		\textsc{Se comunica con} & \multicolumn{1}{p{15.0cm}|}{Sistema de predicción}\\
		\hline

		\textsc{Conocimiento} & \multicolumn{1}{p{15.0cm}|}{El conocimiento que tiene sobre el proceso es elevado}\\
		\hline

		\textsc{Otras competencias} & \multicolumn{1}{p{15.0cm}|}{-}\\
		\hline

		\textsc{Responsabilidades y restricciones} & \multicolumn{1}{p{15.0cm}|}{Maximizar ganancias y minimizar pérdidas}\\
		\hline

	  \end{tabular}
	}
	\caption{\label{tab:AM}Formulario AM-1: Analisis de agentes de la Tarea 5 (Agente: Experto en especulación mercado activos)}
  \end{table}

  \begin{table}[H]
	\centering
	\resizebox{15.0cm}{!}{
	  \begin{tabular}{|l|l|} 
		\hline
		\textbf{Modelo de Agentes} & \textbf{Formulario TM-1: Agente}\\ 
		\hline\hline
		\textsc{Nombre} & \multicolumn{1}{p{15.0cm}|}{Sistema inteligente de predicción} \\
		\hline

		\textsc{Organización} & \multicolumn{1}{p{15.0cm}|}{Departamento de Análisis}\\
		\hline

		\textsc{Implicado en} & \multicolumn{1}{p{15.0cm}|}{Tareas 1, 3, 4 y 5}\\
		\hline

		\textsc{Se comunica con} & \multicolumn{1}{p{15.0cm}|}{Otros agentes (Experto en especulación mercado activos)}\\
		\hline

		\textsc{Conocimiento} & \multicolumn{1}{p{15.0cm}|}{El conocimiento sobre el proceso de predecir los precios es elevado}\\
		\hline

		\textsc{Otras competencias} & \multicolumn{1}{p{15.0cm}|}{-}\\
		\hline

		\textsc{Responsabilidades y restricciones} & \multicolumn{1}{p{15.0cm}|}{Es responsable de indicar precios futuros de los activos que analiza y como restricción tenemos que ese agente tiene que poseer una gran cantidad de información sobre el activo en cuestión}\\
		\hline

	  \end{tabular}
	}
	\caption{\label{tab:AM2}Formulario AM-1: Analisis de agentes de la Tarea 5 (Agente: Sistema inteligente de predicción)}
  \end{table}




