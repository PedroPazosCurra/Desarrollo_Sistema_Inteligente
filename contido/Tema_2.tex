%%%%%%%%%%%%%%%%%%%%%%%%%%%%%%%%%%%%%%%%%%%%%%%%%%%%%%%%%%%%%%%%%%%%%%%
% Plantilla TFG/TFM
% Universidad de A Coruña. Facultad de Informática
% Realizado por: Welton Vieira dos Santos
% Modificado: Welton Vieira dos Santos
% Contacto: welton.dossantos@udc.es
%%%%%%%%%%%%%%%%%%%%%%%%%%%%%%%%%%%%%%%%%%%%%%%%%%%%%%%%%%%%%%%%%%%%%%%%


\chapter{Análisis de Impactos y Mejoras: Modelado de las Tareas y los Agentes.}
\newpage
\section{Formulario TM-1: análisis de tareas.}
Descripción detallada de tareas en el contexto del proceso de interés.

% Formulario TM-1 de la tarea 1
\begin{table}[H]
	\scriptsize
  	\resizebox{14,2cm}{!}{
		\begin{tabularx}{\textwidth}{|l|X|} 
			\hline
			\textbf{Modelo de Tareas} & \textbf{Formulario TM-1: Análisis de Tareas} \\ 
			\hline
			\hline

			\textsc{Tarea} & \textbf{Tarea 1:} \textit{Recibimiento de los paquetes en la sucursal MRW (entrada)} \\ 
			\hline
			\textsc{Organización}  & \textit{Logística.} \\ 
			\hline
			\textsc{Objetivo y valor} &  \textit{Es una parte obligatoria de todo el proceso.} \\ 
			\hline
			\textsc{Dependencia y Flujos} & 
				\begin{enumerate}
					\item \textbf{Tareas precedentes:} \textit{Ninguna}
					\item \textbf{Tareas que le siguen:} \textit{Tarea 2}
				\end{enumerate} \\
			\hline
			\textsc{Objetos manipulados} & 
				\begin{enumerate}
					\item \textbf{Objetos de entrada de la tarea:} \textit{Información de los datos de los paquetes (a través de las herramientas de lectura de códigos de barras o códigos QR).}
					\item \textbf{Objetos de salida de la tarea:} \textit{Confirmación de recepción del paquete.}
					\item \textbf{Objetos internos:} \textit{Almacena la información del paquete de entrada en una base de datos de paquetería de la sucursal.} 
				\end{enumerate} \\ 
			\hline
			\textsc{Tiempo y control} & 
				\begin{enumerate}
					\item \textbf{Frecuencia y duración:} \textit{Es una tarea que se realiza cada vez que se da la entrada de nueva paquetería y la duración es instantánea}
					\item \textbf{Control:} \textit{Con respecto a otras  tareas, ninguno.}
					\item \textbf{Restricciones:} \textit{Se necesita conexión constante con un sistema perteneciente a MRW para la consulta de la información de cada paquete.}
				\end{enumerate} \\
			\hline
			\textsc{Agentes} & 
				\begin{enumerate}
					\item \textbf{Repartidor:} \textit{Descargue los paquetes del transporte (furgoneta o camión).}
					\item \textbf{Lector de etiquetas:} \textit{Utilizado para identificar la entrada de paquetería en la sucursal.}
				\end{enumerate} \\
			\hline
			\textsc{Conocimiento y Capacidad} & \textit{No necesita ningún conocimiento específico.} \\
			\hline
			\textsc{Recursos} & 
				\begin{itemize}
					\item \textit{Identificador de etiquetas.}
					\item \textit{Mano de obra para efectuar la descarga de la paquetería.}
					\item \textit{Sistema de gestión de control de MRW (base de datos de la paquetería).}
				\end{itemize} \\
			\hline
			\textsc{Calidad y eficiencia} & \textit{Seguir unas pautas para descargar la paquetería, es decir, hay determinados paquetes especiales que tiene un trato diferenciado.} \\
			\hline
		\end{tabularx}
	}
	\caption{\label{tab:TM1T1}Formulario TM-1: Análisis de tarea 1 del OM-3}
\end{table} 

% Formulario TM-1 de la tarea 2
\begin{table}[H]
	\scriptsize
  	\resizebox{14,2cm}{!}{
		\begin{tabularx}{\textwidth}{|l|X|} 
			\hline
			\textbf{Modelo de Tareas} & \textbf{Formulario TM-1: Análisis de Tareas} \\ 
			\hline
			\hline

			\textsc{Tarea} & \textbf{Tarea 2:} \textit{Generar lista de entregas según ruta asignada.} \\ 
			\hline
			\textsc{Organización}  & \textit{Logística.} \\ 
			\hline
			\textsc{Objetivo y valor} &  \textit{Es una parte obligatoria de todo el proceso.} \\ 
			\hline
			\textsc{Dependencia y Flujos} & 
				\begin{enumerate}
					\item \textbf{Tareas precedentes:} \textit{Tarea 1}
					\item \textbf{Tareas que le siguen:} \textit{Tarea 3}
				\end{enumerate} \\
			\hline
			\textsc{Objetos manipulados} & 
				\begin{enumerate}
					\item \textbf{Objetos de entrada de la tarea:} \textit{Información en la etiqueta del paquete.}
					\item \textbf{Objetos de salida de la tarea:} \textit{Información de la ruta que pertenece el paquete.}
					\item \textbf{Objetos internos:} \textit{Hoja de Excel preparada para dejar constancia de los paquetes de cada ruta.} 
				\end{enumerate} 
				\emph{El objeto de salida incluye elementos de conocimiento por parte del repartidor experimentado y, en caso de duda, se consulta el equipo directivo.}\\				
			\hline
			\textsc{Tiempo y control} & 
				\begin{enumerate}
					\item \textbf{Frecuencia y duración:} \textit{Es una tarea que se realiza a la primera hora de cada jornada. La duración es de una hora (aproximadamente).}
					\item \textbf{Control:} \textit{Cada paquete tiene que haber sido procesado en la tarea 1 obligatoriamente.}
					\item \textbf{Restricciones:} \textit{Ninguna restricción, a excepción de la mencionada en 'Control'.}
				\end{enumerate} \\
			\hline
			\textsc{Agentes} & 
				\begin{enumerate}
					\item \textbf{Repartidores/Equipo directivo:} \textit{Van apuntando en una hoja de excel los paquetes que pertenece a cada ruta.}
				\end{enumerate} \\
			\hline
			\textsc{Conocimiento y Capacidad} & \textit{Experiencia por parte del repartidor o, en su defecto, el equipo directivo.} \\
			\hline
			\textsc{Recursos} & 
				\begin{itemize}
					\item \textit{Ordenador con una hoja de Excel preparada para almacenar los paquetes de cada ruta.}
					\item \textit{Acceso a la base de datos de paquetería de MRW.}
					\item \textit{Impresora para imprimir la lista de cada ruta.}
				\end{itemize} \\
			\hline
			\textsc{Calidad y eficiencia} & \textit{Seguir unas pautas de procesado con cada paquete, es decir, incluir todo paquete en la hoja de excel de la ruta asignada correspondiente.} \\
			\hline
		\end{tabularx}
	}
	\caption{\label{tab:TM1T2}Formulario TM-1: Análisis de tarea 2 del OM-3}
\end{table} 

% Formulario TM-1 de la tarea 3
\begin{table}[H]
	\scriptsize
  	\resizebox{14,2cm}{!}{
		\begin{tabularx}{\textwidth}{|l|X|} 
			\hline
			\textbf{Modelo de Tareas} & \textbf{Formulario TM-1: Análisis de Tareas} \\ 
			\hline
			\hline

			\textsc{Tarea} & \textbf{Tarea 3:} \textit{Determinar los recursos disponibles por la sucursal MRW para entregar según ruta.} \\ 
			\hline
			\textsc{Organización}  & \textit{Logística.} \\ 
			\hline
			\textsc{Objetivo y valor} &  \textit{Es una parte obligatoria de todo el proceso.} \\ 
			\hline
			\textsc{Dependencia y Flujos} & 
				\begin{enumerate}
					\item \textbf{Tareas precedentes:} \textit{Tarea 2}
					\item \textbf{Tareas que le siguen:} \textit{Tarea 4}
				\end{enumerate} \\
			\hline
			\textsc{Objetos manipulados} & 
				\begin{enumerate}
					\item \textbf{Objetos de entrada de la tarea:} \textit{Lista de paquetes según la ruta asignada.}
					\item \textbf{Objetos de salida de la tarea:} \textit{Lista ordenada de paquetes según ruta y el vehículo asignado.}
					\item \textbf{Objetos internos:} \textit{conocimiento de los expertos en distribuir los paquetes en el vehículo disponible según el tamaño, carga y tipo de paquetería.} 
				\end{enumerate} 
				\emph{Todos estos objetos incluyen elementos de información y conocimiento.}\\				
			\hline
			\textsc{Tiempo y control} & 
				\begin{enumerate}
					\item \textbf{Frecuencia y duración:} \textit{Es una tarea que se realiza a la primera hora de cada jornada. La duración es de una hora (aproximadamente).}
					\item \textbf{Control:} \textit{Comprobar que la lista ordenada se ha creado correctamente y que ésta tiene sentido (distribución equilibrada de cantidad, tamaño y carga en los vehículos, procurar cercanía entre los puntos de entrega de cada vehículo, etc.).}
					\item \textbf{Restricciones:} \textit{Obrigatoriedad de la ejecucuón de las tareas 1 y 2 previamente.}
				\end{enumerate} \\
			\hline
			\textsc{Agentes} & 
				\begin{enumerate}
					\item \textbf{Repartidor/Equipo directivo:} \textit{Buscan la mejor forma de adaptar la carga a los vehículos de transporte.}
					\item \textbf{Sistema de gestión de paquetes de MRW:} \textit{Permite hacer una consulta en busca de la información del paquete.}
				\end{enumerate} \\
			\hline
			\textsc{Conocimiento y Capacidad} & \textit{Experiencia por parte del repartidor o en su caso, el equipo directivo.} \\
			\hline
			\textsc{Recursos} & 
				\begin{itemize}
					\item \textit{Tiempo es el recurso primordial en esa tarea.}
				\end{itemize} \\
			\hline
			\textsc{Calidad y eficiencia} & \textit{Seguir unas pautas de colocación de los paquetes en los vehículos de transporte, la calidad se delega en la tarea 4.} \\
			\hline
		\end{tabularx}
	}
	\caption{\label{tab:TM1T3}Formulario TM-1: Análisis de tarea 3 del OM-3}
\end{table} 

% Formulario TM-1 de la tarea 4
\begin{table}[H]
	\scriptsize
  	\resizebox{14,2cm}{!}{
		\begin{tabularx}{\textwidth}{|l|X|} 
			\hline
			\textbf{Modelo de Tareas} & \textbf{Formulario TM-1: Análisis de Tareas} \\ 
			\hline
			\hline

			\textsc{Tarea} & \textbf{Tarea 4:} \textit{Revisión y validación de la distribución de la paquetería.} \\ 
			\hline
			\textsc{Organización}  & \textit{Logística.} \\ 
			\hline
			\textsc{Objetivo y valor} &  \textit{Es una parte obligatoria de todo el proceso.} \\ 
			\hline
			\textsc{Dependencia y Flujos} & 
				\begin{enumerate}
					\item \textbf{Tareas precedentes:} \textit{Tarea 3}
					\item \textbf{Tareas que le siguen:} \textit{Ninguna}
				\end{enumerate} \\
			\hline
			\textsc{Objetos manipulados} & 
				\begin{enumerate}
					\item \textbf{Objetos de entrada de la tarea:} \textit{Lista ordenada de paquetes según ruta y el vehículo asignado.}
					\item \textbf{Objetos de salida de la tarea:} \textit{Una validación de la lista ordenada de paquetes según ruta y el vehículo asignado.}
					\item \textbf{Objetos internos:} \textit{Conocimiento de los expertos en distribuir paquetes en los vehículos disponibles según el tamaño, carga y tipo de paquetería.} 
				\end{enumerate} 
				\emph{Todos estos objetos incluyen elementos de información y conocimiento.}\\				
			\hline
			\textsc{Tiempo y control} & 
				\begin{enumerate}
					\item \textbf{Frecuencia y duración:} \textit{Es una tarea que se realiza a primera hora de cada jornada. La duración es de una hora (aproximadamente).}
					\item \textbf{Control:} \textit{Comprobar que la lista ordenada se ha creado correctamente y que ésta tiene sentido (distribución equilibrada de cantidad, tamaño y carga en los vehículos, procurar cercanía entre los puntos de entrega de cada vehículo, etc.).}
					\item \textbf{Restricciones:} \textit{Obrigatoriedad de la ejecucuón de las tareas 1, 2 y 3 previamente.}
				\end{enumerate} \\
			\hline
			\textsc{Agentes} & 
				\begin{enumerate}
					\item \textbf{Equipo directivo:} \textit{Buscan la mejor forma de adaptar la carga a los vehículos de transporte.}
					\item \textbf{Sistema de gestión de paquetes de MRW:} \textit{Permite hacer una consulta en busca de la información del paquete.}
				\end{enumerate} \\
			\hline
			\textsc{Conocimiento y Capacidad} & \textit{Experiencia por parte del equipo directivo.} \\
			\hline
			\textsc{Recursos} & 
				\begin{itemize}
					\item \textit{Tiempo es el recurso primordial en esa tarea.}
				\end{itemize} \\
			\hline
			\textsc{Calidad y eficiencia} & \textit{Seguir unas pautas de colocación de paquetes en los vehículos de transporte, la calidad esta basada en la decisión del equipo directivo.} \\
			\hline
		\end{tabularx}
	}
	\caption{\label{tab:TM1T4}Formulario TM-1: Análisis de tarea 4 del OM-3}
\end{table} 
\newpage

\section{Formulario TM-2: Análisis de los cuellos de botella del conocimiento.}
Especificación del conocimiento que se emplea en una tarea, sus cuellos de botella y posibles mejoras.

%% TM-2  - Tarea 2 %%

\begin{table}[H]
	\centering
	\resizebox{15.0cm}{!}{
	  \begin{tabular}{|l|l|l|} 
		\hline
		\textbf{Modelo de Tareas} & \multicolumn{2}{p{15.0cm}|}{\textbf{Formulario TM-2: Elementos de conocimiento}}\\ 
		\hline\hline

		\textsc{Nombre} & \multicolumn{2}{p{15.0cm}|}{Experiencia en reparto de paquetes según la ruta
		asignada} \\
		\hline

		\textsc{Poseído por} & \multicolumn{2}{p{15.0cm}|}{Expertos en reparto} \\
		\hline

		\textsc{Usado en} & \multicolumn{2}{p{15.0cm}|}{Tarea 2 - Generar lista de entregas según ruta asignada} \\
		\hline

		\textsc{Dominio} & \multicolumn{2}{p{15.0cm}|}{Experiencia en reparto} \\
		\hline

		\textsc{\textbf{Naturaleza del conocimiento}} & \multicolumn{1}{p{1.2cm}|}{\centering \textit{\textbf{Si/No}}} & \multicolumn{1}{p{13.0cm}|}{\textbf{¿Cuello de botella/debe ser mejorado?}}\\
		\hline

		Formal, riguroso & \multicolumn{1}{p{1.0cm}|}{No} & \multicolumn{1}{p{13.0cm}|}{No}\\
		\hline

		Empírico, cuantitativo & \multicolumn{1}{p{1.0cm}|}{No} & \multicolumn{1}{p{13.0cm}|}{No}\\
		\hline

		Heurístico, sentido común & \multicolumn{1}{p{1.0cm}|}{Sí} & \multicolumn{1}{p{13.0cm}|}{Sí es mejorable, no es fácil de transferir}\\
		\hline

		\multicolumn{1}{|p{6.0cm}|}{Altamente especializado, específico del dominio} & \multicolumn{1}{p{1.0cm}|}{No} & \multicolumn{1}{p{13.0cm}|}{No}\\
		\hline

		Basado en la experiencia & \multicolumn{1}{p{1.0cm}|}{Si} & \multicolumn{1}{p{13.0cm}|}{Si, demasiada dependencia innecesaria en la experiencia}\\
		\hline

		Basado en la acción & \multicolumn{1}{p{1.0cm}|}{No} & \multicolumn{1}{p{13.0cm}|}{No}\\
		\hline

		Incompleto & \multicolumn{1}{p{1.0cm}|}{Sí} & \multicolumn{1}{p{13.0cm}|}{Sí, debe permitir inferir e inducir una forma de actuar a todas las situaciones posibles}\\
		\hline

		Incierto, puede ser incorrecto & \multicolumn{1}{p{1.0cm}|}{Sí} & \multicolumn{1}{p{13.0cm}|}{Sí}\\
		\hline

		Cambia con rapidez & \multicolumn{1}{p{1.0cm}|}{No} & \multicolumn{1}{p{13.0cm}|}{No}\\
		\hline

		Dificil de verificar & \multicolumn{1}{p{1.0cm}|}{No} & \multicolumn{1}{p{13.0cm}|}{No}\\
		\hline

		Tácito, dificil de transferir& \multicolumn{1}{p{1.0cm}|}{Sí} & \multicolumn{1}{p{13.0cm}|}{Sí, no es fácil de transferir}\\
		\hline

		\textsc {\textbf{Forma del conocimiento}}& \multicolumn{1}{p{1.0cm}|}{} & \multicolumn{1}{p{13.0cm}|}{}\\
		\hline

		Mental & \multicolumn{1}{p{1.0cm}|}{Si} & \multicolumn{1}{p{13.0cm}|}{Sí, debemos pasar a un medio que nos permita trabajar con él}\\
		\hline

		Papel & \multicolumn{1}{p{1.0cm}|}{No} & \multicolumn{1}{p{13.0cm}|}{No}\\
		\hline

		Electrónica & \multicolumn{1}{p{1.0cm}|}{No} & \multicolumn{1}{p{13.0cm}|}{No}\\
		\hline

		Habilidades & \multicolumn{1}{p{1.0cm}|}{No} & \multicolumn{1}{p{13.0cm}|}{No}\\
		\hline

		Otros & \multicolumn{1}{p{1.0cm}|}{No} & \multicolumn{1}{p{13.0cm}|}{No}\\
		\hline

		\textsc {\textbf{Disponibilidad del Conocimiento}} & \multicolumn{1}{p{1.0cm}|}{} & \multicolumn{1}{p{13.0cm}|}{}\\
		\hline
		Limitaciones de tiempo& \multicolumn{1}{p{1.0cm}|}{Sí} & \multicolumn{1}{p{13.0cm}|}{Sí, dependemos de los expertos}\\
		\hline

		Limitaciones de espacio& \multicolumn{1}{p{1.0cm}|}{No} & \multicolumn{1}{p{13.0cm}|}{No}\\
		\hline

		Limitaciones de acceso& \multicolumn{1}{p{1.0cm}|}{No} & \multicolumn{1}{p{13.0cm}|}{No}\\
		\hline

		Limitaciones de calidad& \multicolumn{1}{p{1.0cm}|}{Sí} & \multicolumn{1}{p{13.0cm}|}{Sí, depende de la calidad de los expertos}\\
		\hline

		Limitaciones de forma& \multicolumn{1}{p{1.0cm}|}{No} & \multicolumn{1}{p{13.0cm}|}{No}\\
		\hline

	  \end{tabular}
	}
	\caption{\label{tab:TM2}Formulario TM-2: Analisis de cuellos de botella en la Tarea 2 - (Experiencia en reparto de paquetes...)}
  \end{table}

%% TM-2 (2) %%

  \begin{table}[H]
	\centering
	\resizebox{15.0cm}{!}{
	  \begin{tabular}{|l|l|l|} 
		\hline
		\textbf{Modelo de Tareas} & \multicolumn{2}{p{15.0cm}|}{\textbf{Formulario TM-2: Elementos de conocimiento}}\\ 
		\hline\hline

		\textsc{Nombre} & \multicolumn{2}{p{15.0cm}|}{Experiencia en
		distribución de recursos} \\
		\hline

		\textsc{Poseído por} & \multicolumn{2}{p{15.0cm}|}{Expertos (director) y equipo directivo)} \\
		\hline

		\textsc{Usado en} & \multicolumn{2}{p{15.0cm}|}{Tarea 3 (Determinar los recursos...) y 4 (Revisión y validación de la distribución de la paquetería) de OM-3} \\
		\hline

		\textsc{Dominio} & \multicolumn{2}{p{15.0cm}|}{Experiencia} \\
		\hline

		\textsc{\textbf{Naturaleza del conocimiento}} & \multicolumn{1}{p{1.2cm}|}{\centering \textit{\textbf{Si/No}}} & \multicolumn{1}{p{13.0cm}|}{\textbf{¿Cuello de botella/debe ser mejorado?}}\\
		\hline

		Formal, riguroso & \multicolumn{1}{p{1.0cm}|}{No} & \multicolumn{1}{p{13.0cm}|}{Sí, no debe depender de la experiencia}\\
		\hline

		Empírico, cuantitativo & \multicolumn{1}{p{1.0cm}|}{Sí} & \multicolumn{1}{p{13.0cm}|}{No}\\
		\hline

		Heurístico, sentido común & \multicolumn{1}{p{1.0cm}|}{Sí} & \multicolumn{1}{p{13.0cm}|}{Sí, porque hay que tener experiencia a la hora de elegir la mejor solución}\\
		\hline

		\multicolumn{1}{|p{6.0cm}|}{Altamente especializado, específico del dominio} & \multicolumn{1}{p{1.0cm}|}{Si} & \multicolumn{1}{p{13.0cm}|}{Sí, se necesitan conocimientos del dominio en concreto, no se puede abstraer}\\
		\hline

		Basado en la experiencia & \multicolumn{1}{p{1.0cm}|}{Sí} & \multicolumn{1}{p{13.0cm}|}{Sí, depende de la experiencia del experto.}\\
		\hline

		Basado en la acción & \multicolumn{1}{p{1.0cm}|}{No} & \multicolumn{1}{p{13.0cm}|}{No}\\
		\hline

		Incompleto & \multicolumn{1}{p{1.0cm}|}{No} & \multicolumn{1}{p{13.0cm}|}{No}\\
		\hline

		Incierto, puede ser incorrecto & \multicolumn{1}{p{1.0cm}|}{Sí} & \multicolumn{1}{p{13.0cm}|}{Sí. Aunque sabemos que parte parte del personal directivo posee conocimientos de programación dinámica, existen miembros del equipo que no. Estos últimos hacen uso de su experiencia y capacidad de raciocinio, y pueden cometer errores.}\\
		\hline

		Cambia con rapidez & \multicolumn{1}{p{1.0cm}|}{No} & \multicolumn{1}{p{13.0cm}|}{No}\\
		\hline

		Dificil de verificar & \multicolumn{1}{p{1.0cm}|}{Sí} & \multicolumn{1}{p{13.0cm}|}{Sí}\\
		\hline

		Tácito, dificil de transferir& \multicolumn{1}{p{1.0cm}|}{Sí} & \multicolumn{1}{p{13.0cm}|}{Sí, debe estar formalizado y validado}\\
		\hline

		\textsc {\textbf{Forma del conocimiento}}& \multicolumn{1}{p{1.0cm}|}{} & \multicolumn{1}{p{13.0cm}|}{}\\
		\hline

		Mental & \multicolumn{1}{p{1.0cm}|}{Sí} & \multicolumn{1}{p{13.0cm}|}{Sí, debe formalizarse}\\
		\hline

		Papel & \multicolumn{1}{p{1.0cm}|}{No} & \multicolumn{1}{p{13.0cm}|}{No}\\
		\hline

		Electrónica & \multicolumn{1}{p{1.0cm}|}{No} & \multicolumn{1}{p{13.0cm}|}{No}\\
		\hline

		Habilidades & \multicolumn{1}{p{1.0cm}|}{No} & \multicolumn{1}{p{13.0cm}|}{No}\\
		\hline

		Otros & \multicolumn{1}{p{1.0cm}|}{No} & \multicolumn{1}{p{13.0cm}|}{No}\\
		\hline

		\textsc {\textbf{Disponibilidad del Conocimiento}} & \multicolumn{1}{p{1.0cm}|}{} & \multicolumn{1}{p{13.0cm}|}{}\\
		\hline
		Limitaciones de tiempo& \multicolumn{1}{p{1.0cm}|}{No} & \multicolumn{1}{p{13.0cm}|}{No}\\
		\hline

		Limitaciones de espacio& \multicolumn{1}{p{1.0cm}|}{No} & \multicolumn{1}{p{13.0cm}|}{No}\\
		\hline

		Limitaciones de acceso& \multicolumn{1}{p{1.0cm}|}{Sí} & \multicolumn{1}{p{13.0cm}|}{Sí, dependemos de los expertos que posean esa información}\\
		\hline

		Limitaciones de calidad& \multicolumn{1}{p{1.0cm}|}{Sí} & \multicolumn{1}{p{13.0cm}|}{Sí, depende del conocimiento del experto.}\\
		\hline

		Limitaciones de forma& \multicolumn{1}{p{1.0cm}|}{Sí} & \multicolumn{1}{p{13.0cm}|}{Sí, debe formalizarse}\\
		\hline

	  \end{tabular}
	}
	\caption{\label{tab:TM22}Formulario TM-2: Análisis de cuellos de botella en la Tarea 3 (Experiencia en distribución de recursos...) y 4 (Revisión y validación de la distribución de la paquetería)}
  \end{table}
  \newpage

	% TM-2 (3)

  \begin{table}[H]
	\centering
	\resizebox{15.0cm}{!}{
	  \begin{tabular}{|l|l|l|} 
		\hline
		\textbf{Modelo de Tareas} & \multicolumn{2}{p{15.0cm}|}{\textbf{Formulario TM-2: Elementos de conocimiento}}\\ 
		\hline\hline

		\textsc{Nombre} & \multicolumn{2}{p{15.0cm}|}{Utilizar teoría de
		programación dinámica} \\
		\hline

		\textsc{Poseído por} & \multicolumn{2}{p{15.0cm}|}{Expertos (director) y equipo directivo)} \\
		\hline

		\textsc{Usado en} & \multicolumn{2}{p{15.0cm}|}{Tarea 3 (Determinar los recursos...) y 4 (Revisión y validación de la distribución de la paquetería) de OM-3} \\
		\hline

		\textsc{Dominio} & \multicolumn{2}{p{15.0cm}|}{Matemática discreta} \\
		\hline

		\textsc{\textbf{Naturaleza del conocimiento}} & \multicolumn{1}{p{1.2cm}|}{\centering \textit{\textbf{Si/No}}} & \multicolumn{1}{p{13.0cm}|}{\textbf{¿Cuello de botella/debe ser mejorado?}}\\
		\hline

		Formal, riguroso & \multicolumn{1}{p{1.0cm}|}{Sí} & \multicolumn{1}{p{13.0cm}|}{Sí}\\
		\hline

		Empírico, cuantitativo & \multicolumn{1}{p{1.0cm}|}{Sí} & \multicolumn{1}{p{13.0cm}|}{Sí}\\
		\hline

		Heurístico, sentido común & \multicolumn{1}{p{1.0cm}|}{No} & \multicolumn{1}{p{13.0cm}|}{No}\\
		\hline

		\multicolumn{1}{|p{6.0cm}|}{Altamente especializado, específico del dominio} & \multicolumn{1}{p{1.0cm}|}{Sí} & \multicolumn{1}{p{13.0cm}|}{No}\\
		\hline

		Basado en la experiencia & \multicolumn{1}{p{1.0cm}|}{No} & \multicolumn{1}{p{13.0cm}|}{No}\\
		\hline

		Basado en la acción & \multicolumn{1}{p{1.0cm}|}{No} & \multicolumn{1}{p{13.0cm}|}{No}\\
		\hline

		Incompleto & \multicolumn{1}{p{1.0cm}|}{No} & \multicolumn{1}{p{13.0cm}|}{No}\\
		\hline

		Incierto, puede ser incorrecto & \multicolumn{1}{p{1.0cm}|}{No} & \multicolumn{1}{p{13.0cm}|}{No}\\
		\hline

		Cambia con rapidez & \multicolumn{1}{p{1.0cm}|}{No} & \multicolumn{1}{p{13.0cm}|}{No}\\
		\hline

		Dificil de verificar & \multicolumn{1}{p{1.0cm}|}{No} & \multicolumn{1}{p{13.0cm}|}{No}\\
		\hline

		Tácito, dificil de transferir& \multicolumn{1}{p{1.0cm}|}{No} & \multicolumn{1}{p{13.0cm}|}{No}\\
		\hline

		\textsc {\textbf{Forma del conocimiento}}& \multicolumn{1}{p{1.0cm}|}{} & \multicolumn{1}{p{13.0cm}|}{}\\
		\hline

		Mental & \multicolumn{1}{p{1.0cm}|}{No} & \multicolumn{1}{p{13.0cm}|}{No}\\
		\hline

		Papel & \multicolumn{1}{p{1.0cm}|}{Sí} & \multicolumn{1}{p{13.0cm}|}{No}\\
		\hline

		Electrónica & \multicolumn{1}{p{1.0cm}|}{No} & \multicolumn{1}{p{13.0cm}|}{No}\\
		\hline

		Habilidades & \multicolumn{1}{p{1.0cm}|}{No} & \multicolumn{1}{p{13.0cm}|}{No}\\
		\hline

		Otros & \multicolumn{1}{p{1.0cm}|}{No} & \multicolumn{1}{p{13.0cm}|}{No}\\
		\hline

		\textsc {\textbf{Disponibilidad del Conocimiento}} & \multicolumn{1}{p{1.0cm}|}{} & \multicolumn{1}{p{13.0cm}|}{}\\
		\hline
		Limitaciones de tiempo& \multicolumn{1}{p{1.0cm}|}{No} & \multicolumn{1}{p{13.0cm}|}{No}\\
		\hline

		Limitaciones de espacio& \multicolumn{1}{p{1.0cm}|}{No} & \multicolumn{1}{p{13.0cm}|}{No}\\
		\hline

		Limitaciones de acceso& \multicolumn{1}{p{1.0cm}|}{No} & \multicolumn{1}{p{13.0cm}|}{No}\\
		\hline

		Limitaciones de calidad& \multicolumn{1}{p{1.0cm}|}{No} & \multicolumn{1}{p{13.0cm}|}{No}\\
		\hline

		Limitaciones de forma& \multicolumn{1}{p{1.0cm}|}{No} & \multicolumn{1}{p{13.0cm}|}{No}\\
		\hline

	  \end{tabular}
	}
	\caption{\label{tab:TM23}Formulario TM-2: Análisis de cuellos de botella en la Tarea 3 (Experiencia en distribución de recursos...) y 4 (Revisión y validación de la distribución de la paquetería)}
  \end{table}

\newpage

%%% AM-1 (Descripción de los agentes) %%%

\section{Formulario AM-1: descripción de los agentes.}
Descripción de los agentes implicados en las tareas de interés.

\begin{table}[H]
	\centering
	\resizebox{15.0cm}{!}{
	  \begin{tabular}{|l|l|} 
		\hline
		\textbf{Modelo de Agentes} & \textbf{Formulario AM-1: Agente}\\ 
		\hline\hline
		\textsc{Nombre} & \multicolumn{1}{p{15.0cm}|}{Experto en reparto de paquetería} \\
		\hline

		\textsc{Organización} & \multicolumn{1}{p{15.0cm}|}{Departamento de logística}\\
		\hline

		\textsc{Implicado en} & \multicolumn{1}{p{15.0cm}|}{Tareas 1, 2 y 3}\\
		\hline

		\textsc{Se comunica con} & \multicolumn{1}{p{15.0cm}|}{Sistema de gestión de paquetería de MRW y equipo directivo}\\
		\hline

		\textsc{Conocimiento} & \multicolumn{1}{p{15.0cm}|}{El conocimiento que tiene sobre el proceso es elevado}\\
		\hline

		\textsc{Otras competencias} & \multicolumn{1}{p{15.0cm}|}{-}\\
		\hline

		\textsc{Responsabilidades y restricciones} & \multicolumn{1}{p{15.0cm}|}{Seleccionar bien las rutas a las que pertenece cada paquete y asegurarse de maximizar la carga en el transporte}\\
		\hline

	  \end{tabular}
	}
	\caption{\label{tab:AM}Formulario AM-1: Analisis del agente: Experto en reparto de paquetería}
  \end{table}

  \begin{table}[H]
	\centering
	\resizebox{15.0cm}{!}{
	  \begin{tabular}{|l|l|} 
		\hline
		\textbf{Modelo de Agentes} & \textbf{Formulario AM-1: Agente}\\ 
		\hline\hline
		\textsc{Nombre} & \multicolumn{1}{p{15.0cm}|}{Equipo directivo} \\
		\hline

		\textsc{Organización} & \multicolumn{1}{p{15.0cm}|}{Departamento de dirección}\\
		\hline

		\textsc{Implicado en} & \multicolumn{1}{p{15.0cm}|}{Tareas 3 y 4}\\
		\hline

		\textsc{Se comunica con} & \multicolumn{1}{p{15.0cm}|}{Otros agentes (Expertos en reparto de paquetería)}\\
		\hline

		\textsc{Conocimiento} & \multicolumn{1}{p{15.0cm}|}{El conocimiento sobre el proceso es elevado}\\
		\hline

		\textsc{Otras competencias} & \multicolumn{1}{p{15.0cm}|}{Encargado de dirigir el resto de departamentos}\\
		\hline

		\textsc{Responsabilidades y restricciones} & \multicolumn{1}{p{15.0cm}|}{Es responsable de generar las rutas, organizar los recursos y gestionar el resto de departamentos de la empresa}\\
		\hline

	  \end{tabular}
	}
	\caption{\label{tab:AM21}Formulario AM-1: Analisis del agente: Equipo directivo}
  \end{table}

  \begin{table}[H]
	\centering
	\resizebox{15.0cm}{!}{
	  \begin{tabular}{|l|l|} 
		\hline
		\textbf{Modelo de Agentes} & \textbf{Formulario AM-1: Agente}\\ 
		\hline\hline
		\textsc{Nombre} & \multicolumn{1}{p{15.0cm}|}{Sistema de gestión de MRW} \\
		\hline

		\textsc{Organización} & \multicolumn{1}{p{15.0cm}|}{Departamento de logística}\\
		\hline

		\textsc{Implicado en} & \multicolumn{1}{p{15.0cm}|}{Tarea 1}\\
		\hline

		\textsc{Se comunica con} & \multicolumn{1}{p{15.0cm}|}{Otros agentes (Expertos en reparto de paquetería)}\\
		\hline

		\textsc{Conocimiento} & \multicolumn{1}{p{15.0cm}|}{-}\\
		\hline

		\textsc{Otras competencias} & \multicolumn{1}{p{15.0cm}|}{-}\\
		\hline

		\textsc{Responsabilidades y restricciones} & \multicolumn{1}{p{15.0cm}|}{Es responsable de la información de los paquetes recibidos en la nave}\\
		\hline

	  \end{tabular}
	}
	\caption{\label{tab:AM22}Formulario AM-1: Analisis del agente externo: Sistema de gestión MRW}
  \end{table}




