%%%%%%%%%%%%%%%%%%%%%%%%%%%%%%%%%%%%%%%%%%%%%%%%%%%%%%%%%%%%%%%%%%%%%%%%
% Plantilla TFG/TFM
% Universidad de A Coruña. Facultad de Informática
% Realizado por: Welton Vieira dos Santos
% Modificado: Welton Vieira dos Santos
% Contacto: welton.dossantos@udc.es
%%%%%%%%%%%%%%%%%%%%%%%%%%%%%%%%%%%%%%%%%%%%%%%%%%%%%%%%%%%%%%%%%%%%%%%%


\chapter{Modelo de conocimiento}
\newpage
\section{Fase de identificación}
\subsection{Tareas del formulario OM-3}
Las tareas elegidas para este modelo conceptual han sido las tareas 2, 3 y 4 del OM-3 (Tabla \ref{tab:IdentificacionOM3}), que corresponden con \textbf{Generar lista de entregas según ruta asignada}, \textbf{Determinar los recursos disponibles por la sucursal MRW para entregar según ruta} y \textbf{Revisión y validación de la distribución de la paquetería}.

\begin{table}[H]
  \centering
  \resizebox{15,0cm}{!}{
    \begin{tabular}{|c|c|c|c|c|c|c|}
      \hline
      \multicolumn{3}{|c}{\textbf{Modelo de Organización}} & \multicolumn{4}{|c|}{\textbf{Formulario OM-3: Descomposición de los Procesos}}\\
      \hline \hline
      
      \multicolumn{1}{|p{1.0cm}|}{\centering \textsc{N\textordmasculine}} &\multicolumn{1}{|p{3.0cm}|}{\centering \textsc{Tarea}} & \multicolumn{1}{|p{3.0cm}|}{\centering \textsc{Realiza\-da por}} & \multicolumn{1}{|p{3.0cm}|}{\centering \textsc{¿Dónde?}} & \multicolumn{1}{|p{3.0cm}|}{\centering \textsc{Recursos de Conocimiento}} & \multicolumn{1}{|p{3.0cm}|}{\centering \textsc {¿In\-ten\-si\-va en Conocimiento?}} & \multicolumn{1}{|p{3.0cm}|}{\centering \textsc{Im\-por\-tan\-cia}} \\
      \hline

      3 & \multicolumn{1}{|p{3.0cm}|}{\centering Determinar los recursos disponibles por la sucursal MRW para entregar según ruta} & \multicolumn{1}{|p{3.0cm}|}{\centering Equipo Directivo, Repartidor experimentado} & \multicolumn{1}{|p{3.0cm}|}{\centering En la nave de la sucursal de MRW} & \multicolumn{1}{|p{3.0cm}|}{\centering Experiencia en distribución de recursos. Utilizar teoria de programación dinámica} & Sí (elevado) & Máxima \\
      \hline
    \end{tabular}
  }
	\caption{\label{tab:IdentificacionOM3}Tarea elegida para el modelo de conocimiento}
\end{table}

\subsection{Glosario de términos}
\begin{itemize}
	\item \textbf{Paquete:} Envoltorio que contiene cualquier tipo de objeto. Se utiliza tambien para indicar los sobres.
	\item \textbf{Ruta:} Camino conformado por un número determinado de nodos (paradas) con un principio y un final que se cubre con un vehículo de reparto.  
	\item \textbf{Destinatario:} Persona o entidad que recibe un paquete.
	\item \textbf{Remitente:} Persona o entidad que envía o remite a otra persona un paquete.
	\item \textbf{Incidencia:} Circunstancia que impide una entrega.
	\item \textbf{Plataforma:} Local de asignación de envíos.
	\item \textbf{Envío:} Agrupación para uno o mas paquetes de un mismo destinatario.
	\item \textbf{Recogida:} Forma en la que el destinatario recoge el paquete de un remitente.
\end{itemize}

\subsection{Descripción de escenarios}
La empresa actualmente cuenta con tres rutas, conocidas como \textbf{Ruta A, Ruta B} y \textbf{Ruta C}, para efetuar el reparto de la zona según contracto con la central de MRW. Para cubrir esas rutas la organización tiene como recursos de reparto:
\begin{itemize}
  \item 3 furgonetas como se muestra en la Figura \ref{fig:FurgonetaBase}.
  \item 1 furgoneta como se muestra en la Figura \ref{fig:FurgonetaFrio}.
  \item 1 furgoneta como se muestra en la Figura \ref{fig:FurgonetaAnimales}.
\end{itemize}

\begin{figure}[H]
  \centering
  \includegraphics[scale=0.50]{imaxes/FurgonetaBase.png}
  \caption{\label{fig:FurgonetaBase}Ejemplo de la furgoneta base}
\end{figure}

\begin{figure}[H]
  \centering
  \includegraphics[scale=0.50]{imaxes/FurgonetaFrio.png}
  \caption{\label{fig:FurgonetaFrio}Ejemplo de la furgoneta para cargas en frio}
\end{figure}

\begin{figure}[H]
  \centering
  \includegraphics[scale=0.50]{imaxes/FurgonetaAnimales.png}
  \caption{\label{fig:FurgonetaAnimales}Ejemplo de la furgoneta para transportar animales}
\end{figure}

\begin{table}[H]
  \centering
  \resizebox{15,0cm}{!}{
    \begin{tabular}{|c|c|c|c|}
      \hline
      \multicolumn{4}{|c|}{\textbf{Lista de paquetes para asignar recursos para la Ruta A}} \\
      \hline \hline
      
      \multicolumn{1}{|p{1.0cm}|}{\centering \textsc{N\textordmasculine}} &\multicolumn{1}{|p{3.0cm}|}{\centering \textsc{Dimensión (cm)}} & \multicolumn{1}{|p{3.0cm}|}{\centering \textsc{Peso (kg)}} & \multicolumn{1}{|p{3.0cm}|}{\centering \textsc{Tipo de paquete}} \\
      \hline

      1 & \multicolumn{1}{|p{3.0cm}|}{\centering 30x30x30} & \multicolumn{1}{|p{3.0cm}|}{\centering 10} & \multicolumn{1}{|p{3.0cm}|}{\centering normal} \\
      \hline
      2 & \multicolumn{1}{|p{3.0cm}|}{\centering 200x100x40} & \multicolumn{1}{|p{3.0cm}|}{\centering 30} & \multicolumn{1}{|p{3.0cm}|}{\centering normal} \\
      \hline
      3 & \multicolumn{1}{|p{3.0cm}|}{\centering 100x50x40} & \multicolumn{1}{|p{3.0cm}|}{\centering 15} & \multicolumn{1}{|p{3.0cm}|}{\centering frio} \\
      \hline
    \end{tabular}
  }
	\caption{\label{tab:AsignarRecursoRutaA}Lista de paquetes asignados a la Ruta A}
\end{table}

\begin{table}[H]
  \centering
  \resizebox{15,0cm}{!}{
    \begin{tabular}{|c|c|c|c|}
      \hline
      \multicolumn{4}{|c|}{\textbf{Lista de paquetes para asignar recursos para la Ruta B}} \\
      \hline \hline
      
      \multicolumn{1}{|p{1.0cm}|}{\centering \textsc{N\textordmasculine}} &\multicolumn{1}{|p{3.0cm}|}{\centering \textsc{Dimensión (cm)}} & \multicolumn{1}{|p{3.0cm}|}{\centering \textsc{Peso (kg)}} & \multicolumn{1}{|p{3.0cm}|}{\centering \textsc{Tipo de paquete}} \\
      \hline

      1 & \multicolumn{1}{|p{3.0cm}|}{\centering 130x130x130} & \multicolumn{1}{|p{3.0cm}|}{\centering 10} & \multicolumn{1}{|p{3.0cm}|}{\centering normal} \\
      \hline
      
      2 & \multicolumn{1}{|p{3.0cm}|}{\centering 200x200x200} & \multicolumn{1}{|p{3.0cm}|}{\centering 20} & \multicolumn{1}{|p{3.0cm}|}{\centering animal} \\
      \hline
      
      3 & \multicolumn{1}{|p{3.0cm}|}{\centering 100x50x40} & \multicolumn{1}{|p{3.0cm}|}{\centering 15} & \multicolumn{1}{|p{3.0cm}|}{\centering frio} \\
      \hline
    \end{tabular}
  }
	\caption{\label{tab:AsignarRecursoRutaB}Lista de paquetes asignados a la Ruta B}
\end{table}

\begin{table}[H]
  \centering
  \resizebox{15,0cm}{!}{
    \begin{tabular}{|c|c|c|c|}
      \hline
      \multicolumn{4}{|c|}{\textbf{Lista de paquetes para asignar recursos para la Ruta C}} \\
      \hline \hline
      
      \multicolumn{1}{|p{1.0cm}|}{\centering \textsc{N\textordmasculine}} &\multicolumn{1}{|p{3.0cm}|}{\centering \textsc{Dimensión (cm)}} & \multicolumn{1}{|p{3.0cm}|}{\centering \textsc{Peso (kg)}} & \multicolumn{1}{|p{3.0cm}|}{\centering \textsc{Tipo de paquete}} \\
      \hline

      1 & \multicolumn{1}{|p{3.0cm}|}{\centering 130x130x130} & \multicolumn{1}{|p{3.0cm}|}{\centering 10} & \multicolumn{1}{|p{3.0cm}|}{\centering normal} \\
      \hline
      
      2 & \multicolumn{1}{|p{3.0cm}|}{\centering 200x200x200} & \multicolumn{1}{|p{3.0cm}|}{\centering 20} & \multicolumn{1}{|p{3.0cm}|}{\centering animal} \\
      \hline
      
      3 & \multicolumn{1}{|p{3.0cm}|}{\centering 100x50x40} & \multicolumn{1}{|p{3.0cm}|}{\centering 15} & \multicolumn{1}{|p{3.0cm}|}{\centering frio} \\
      \hline
    \end{tabular}
  }
	\caption{\label{tab:AsignarRecursoRutaC}Lista de paquetes asignados a la Ruta C}
\end{table}

Asignar recursos para las siguientes encenarios:
\begin{enumerate}
	\item  \textbf{Listado de carga para la Ruta A:} Asignar recursos para la entrega de la paquetería del listado de la Tabla \ref{tab:AsignarRecursoRutaA}.
	\item  \textbf{Listado de carga para la Ruta B:} Asignar recursos para la entrega de la paquetería del listado de la Tabla \ref{tab:AsignarRecursoRutaB}.
\end{enumerate}

\section{Fase de especificación}
\subsection{Justificación de la selección de la metodología}
Para este proyecto hemos decidido utilizar la metodología ``middle-out'' con la selección de la plantilla de configuración como se muestra en la Figura \ref{fig:PlantillaConfiguracion}, ya que esa plantilla nos permite buscar una mejor distribución de la carga dentro del vehículo asignado a la ruta. 

\begin{figure}[H]
  \centering
  \includegraphics[scale=0.30]{imaxes/PlantillaConfiguracion.png}
  \caption{\label{fig:PlantillaConfiguracion}Ejemplo de la plantilla elegida}
\end{figure}

\subsection{Plantilla anotada}

La plantilla que mas se adapta a nuestro problema es la de configuración, como se muestra en la Figura \ref{fig:PlantillaConfiguracionComentada}.

\begin{figure}[H]
  \centering
  \includegraphics[scale=0.25]{imaxes/PlantillaConfiguracionComentada.png}
  \caption{\label{fig:PlantillaConfiguracionComentada}Ejemplo de la plantilla elegida con las anotaciones pertinentes}
\end{figure}

\newpage

\subsection{Esquema inicial del dominio}

\subsubsection{Plantilla}
La Figura \ref{fig:DiagramaInicialDominio} muestra el diagrama inicial del dominio.

\begin{figure}[H]
  \centering
  \includegraphics[scale=0.50]{imaxes/DiagramaInicialDominio.png}
  \caption{\label{fig:DiagramaInicialDominio}Diagrama inicial del dominio}
\end{figure}

\subsection{Estructura inferencial}

La Figura \ref{fig:EstructuraInferencial} muestra la estructura inferencial de la plantilla elegida.
\begin{figure}[H]
  \centering
  \includegraphics[scale=0.30]{imaxes/EstructuraInferencial.png}
  \caption{\label{fig:EstructuraInferencial}Estructura inferencial de la plantilla elegida}
\end{figure}

\subsubsection{Mapeado}

Relacción entre los roles de las inferencias de la plantilla con los conceptos de nuestro problema.

La Figura \ref{fig:Especificar} muestra el mapeo de la inferencia \textbf{\textit{Especificar}}.

\begin{figure}[H]
  \centering
  \includegraphics[scale=0.35]{imaxes/Especificar.png}
  \caption{\label{fig:Especificar}Mapeado de \textit{especificar}.}
\end{figure}

La Figura \ref{fig:TrasladarPreferibles} muestra el mapeo de la inferencia \textbf{\textit{Trasladar}} a \textit{Requisitos preferibles}.

\begin{figure}[H]
  \centering
  \includegraphics[scale=0.35]{imaxes/TrasladarPreferibles.png}
  \caption{\label{fig:TrasladarPreferibles}Mapeado de \textit{Trasladar} a \textit{Requisitos preferibles}.}
\end{figure}

La Figura \ref{fig:TrasladarObligatorios} muestra el mapeo de la inferencia \textbf{\textit{Trasladar}} a \textit{Requisitos obligatorios}.

\begin{figure}[H]
  \centering
  \includegraphics[scale=0.35]{imaxes/TrasladarObligatorios.png}
  \caption{\label{fig:TrasladarObligatorios}Mapeado de \textit{Trasladar} a \textit{Requisitos obligatorios}.}
\end{figure}

La Figura \ref{fig:PreferiblesProponerExtension} muestra el mapeo de la inferencia \textbf{Proponer} elementos preferibles.

\begin{figure}[H]
  \centering
  \includegraphics[scale=0.35]{imaxes/PreferiblesProponerExtension.png}
  \caption{\label{fig:PreferiblesProponerExtension}Mapeado de \textit{proponer} elementos preferibles.}
\end{figure}
 
La Figura \ref{fig:DisenoEsqueletalProponerExtension} muestra el mapeo de la inferencia \textbf{Proponer} diseño elemental.

\begin{figure}[H]
  \centering
  \includegraphics[scale=0.35]{imaxes/DisenoEsqueletalProponerExtension.png}
  \caption{\label{fig:DisenoEsqueletalProponerExtension}Mapeado de \textit{Proponer} diseño elemental.}
\end{figure}

La Figura \ref{fig:DisenoProponerExtension} muestra el mapeo de la inferencia \textbf{Proponer} diseño actual o modificado.

\begin{figure}[H]
  \centering
  \includegraphics[scale=0.35]{imaxes/DisenoProponerExtension.png}
  \caption{\label{fig:DisenoProponerExtension}Mapeado de \textit{Proponer} diseño actual o modificado.}
\end{figure}

La Figura \ref{fig:DisenoVerificarValor} muestra el mapeo de la inferencia \textbf{Verificar} diseño actual o modificado.

\begin{figure}[H]
  \centering
  \includegraphics[scale=0.35]{imaxes/DisenoVerificarValor.png}
  \caption{\label{fig:DisenoVerificarValor}Mapeado de \textit{Verificar} diseño actual o modificado.}
\end{figure}

La Figura \ref{fig:RequisitosObligatoriosVerificarValor} muestra el mapeo de la inferencia \textbf{Verificar} requisitos obligatorios de diseño.

\begin{figure}[H]
  \centering
  \includegraphics[scale=0.35]{imaxes/RequisitosObligatoriosVerificarValor.png}
  \caption{\label{fig:RequisitosObligatoriosVerificarValor}Mapeado de \textit{Verificar} diseño elemental.}
\end{figure}

La Figura \ref{fig:DisenoVerificarViolacion} muestra el mapeo de la inferencia \textbf{Verificar} la violación de un elemento de diseño.

\begin{figure}[H]
  \centering
  \includegraphics[scale=0.35]{imaxes/DisenoVerificarViolacion.png}
  \caption{\label{fig:DisenoVerificarViolacion}Mapeado de \textit{Verificar} la violación de un elemento de diseño.}
\end{figure}

La Figura \ref{fig:ExtensionVerificarViolacion} muestra el mapeo de la inferencia \textbf{Verificar} la violación de un nuevo elemento de diseño.

\begin{figure}[H]
  \centering
  \includegraphics[scale=0.35]{imaxes/ExtensionVerificarViolacion.png}
  \caption{\label{fig:ExtensionVerificarViolacion}Mapeado de \textit{Verificar} la violación de un nuevo elemento de diseño.}
\end{figure}

La Figura \ref{fig:RequisitosObligatoriosVerificarViolacion} muestra el mapeo de la inferencia \textbf{Verificar} la violación de requisitos obligatorios.

\begin{figure}[H]
  \centering
  \includegraphics[scale=0.35]{imaxes/RequisitosObligatoriosVerificarViolacion.png}
  \caption{\label{fig:RequisitosObligatoriosVerificarViolacion}Mapeado de \textit{Verificar} la violación de requisitos obligatorios.}
\end{figure}

La Figura \ref{fig:DisenoCriticarListaAciones} muestra el mapeo de la inferencia \textbf{Criticar} el diseño.

\begin{figure}[H]
  \centering
  \includegraphics[scale=0.35]{imaxes/DisenoCriticarListaAciones.png}
  \caption{\label{fig:DisenoCriticarListaAciones}Mapeado de \textit{Criticar} el diseño.}
\end{figure}

La Figura \ref{fig:ViolacionCriticarListaAciones} muestra el mapeo de la inferencia \textbf{Criticar} la violación del diseño.

\begin{figure}[H]
  \centering
  \includegraphics[scale=0.35]{imaxes/ViolacionCriticarListaAciones.png}
  \caption{\label{fig:ViolacionCriticarListaAciones}Mapeado de \textit{Criticar} la violación del diseño.}
\end{figure}

La Figura \ref{fig:ListaAccionesSeleccionarAccion} muestra el mapeo de la inferencia \textbf{Seleccionar} acción de la lista de acciones.

\begin{figure}[H]
  \centering
  \includegraphics[scale=0.35]{imaxes/ListaAccionesSeleccionarAccion.png}
  \caption{\label{fig:ListaAccionesSeleccionarAccion}Mapeado de \textit{Seleccionar} accion de la lista de acciones.}
\end{figure}

La Figura \ref{fig:AccionModificaDiseno} muestra el mapeo de la inferencia \textbf{Modificar} diseño.

\begin{figure}[H]
  \centering
  \includegraphics[scale=0.35]{imaxes/AccionModificaDiseno.png}
  \caption{\label{fig:AccionModificaDiseno}Mapeado de \textit{Modificar} diseño.}
\end{figure}

La Figura \ref{fig:DisenoModificarDiseno} muestra el mapeo de la inferencia \textbf{Modificar} diseño desde diseño.

\begin{figure}[H]
  \centering
  \includegraphics[scale=0.35]{imaxes/DisenoModificarDiseno.png}
  \caption{\label{fig:DisenoModificarDiseno}Mapeado de \textit{Modificar} diseño desde diseño.}
\end{figure}


\section{Especificación y método de la tarea}

\subsection{Especificación}

\begin{lstlisting}[language=C,caption=\textbf{Especificación}]
  TASK configuracion-diseño;
    ROLES:
      INPUT: Requisitos: "requisitos para el diseño";
      OUTPUT: Diseño: "el diseño resultante";
  END TASK configuracion-diseño;
\end{lstlisting}

\newpage
\subsection{Método de la Tarea}

\begin{lstlisting}[language=C,caption=\textbf{Método de la tarea}]
  TASK-METHOD proponer-y-revisar;
    REALIZES: configuracion-diseño;
    DECOMPOSITION:
      INFERENCES: Trasladar, Especificar, Proponer, Verificar, Criticar, Seleccionar, Modificar;
    ROLES:
      INTERMEDIATE:
        Requisitos-preferibles: "requisitos que se utilizarán como preferencias (suaves)";
        Requisitos-obligatorios: "requisitos que son restricciones obligatorias (estrictas)";
        Diseño-Esqueletal: "conjunto de elementos de diseño";
        Extensión: "un único valor nuevo para un elemento de diseño";
        Violación: "restricción violada por el diseño actual";
        Valor: "booleano que indica el resultado de la verificación";
        Lista-de-acciones: "lista ordenada de posibles acciones de reparación (fijación)";
        Acción: "una sola acción de reparación";
    CONTROL-STRUCTURE:
      operacionalizar(Requisitos -> Requisitos-preferibles + Requisitos-obligatorios);
      especificar(Requisitos -> Diseño-Esqueletal);
      WHILE NEW-SOLUTION proponer(Diseño-Esqueletal + Diseño + Requisitos-preferibles -> Extensión) DO
        Diseño := Extensión ADD Diseño;
        verificar(Diseño + Requisitos-obligatorios -> Valor + Violación);
        IF Valor == false
        THEN
          criticar(Violación + Diseño -> Lista-de-acciones);
          REPEAT
            seleccionar(Lista-de-acciones -> Acción);
            modificar(Diseño + Acción -> Diseño);
            verificar(Diseño + Requisitos-obligatorios -> Valor + Violación);
          UNTIL Valor == true;
          END REPEAT
        END IF
      END WHILE
  END TASK-METHOD proponer-y-revisar;
\end{lstlisting}

\subsection{Base de conocimiento}

\begin{lstlisting}[language=C,caption=\textbf{Regla\_de\_Traslación}]
  EXPRESSIONS:
    Requisito == Tipo_Requisito_preferible
  RESULTA_EN:
    Requisito_preferible
  END KNOWLEDGE-BASE Regla-translación;
    
  EXPRESSIONS:
    Requisito == Tipo_Requisito_obrigatorio
  RESULTA_EN:
    Requisito_obligatorio
  END KNOWLEDGE-BASE Regla_de_translación;
\end{lstlisting}


\begin{lstlisting}[language=C,caption=\textbf{Regla\_de\_Verificación}]
  EXPRESSIONS:
   Reparto.Fecha == fecha_actual AND Reparto.Repartidor.Cod == 1 AND Reparto.Vehiculo.Cod == 1 AND Reparto.pesototal <= Vehiculo.peso_max AND Parametro.valor == false AND Requisitos_obligatorios == Peso
  RESULTA EN:
   V_peso 
  END KNOWLEDGE-BASE Regla_de_Verificación;
  
  EXPRESSIONS:
   reparto.fecha == fecha_actual and reparto.repartidor.cod == 1 and reparto.vehiculo.cod == 1 and reparto.volumentotal <= vehiculo.volumen_carga and parametro.valor == false and requisitos_obligatorios == Volumen
  RESULTA EN:
   V_Volumen 
  END KNOWLEDGE-BASE Regla_de_Verificación;
\end{lstlisting}
    
\begin{lstlisting}[language=C,caption=\textbf{Regla de abstraccion}]
  EXPRESSIONS:
    accion = AjustarRiesgo AND MercadoForex.Activo.tipo = compra
    RESULTA-EN
         MercadoForex.Activo.EscenarioActivo.Tendencia = alcista

  EXPRESSIONS:
    accion = AjustarRiesgo AND MercadoForex.Activo.tipo = venta
    RESULTA-EN
         MercadoForex.Activo.EscenarioActivo.Tendencia = bajista
  END KNOWLEDGE-BASE regla-especificación;
\end{lstlisting}
      
\begin{lstlisting}[language=C,caption=\textbf{Regla de abstraccion}]
  EXPRESSIONS:
    MercadoForex.Activo.EscenarioActivo.Tendencia = alcista AND Hypothesis = AjustarRiesgo
  RESULTA-EN:
    AjustarRiesgo.evidencia = True
  END KNOWLEDGE-BASE regla-especificación;

  EXPRESSIONS:
    MercadoForex.Activo.EscenarioActivo.Tendencia = bajista AND Hypothesis = AjustarRiesgo
  RESULTA-EN:
    AjustarRiesgo.evidencia = False
  END KNOWLEDGE-BASE regla-especificación;

\end{lstlisting}
\newpage

\subsection{Esquema completo del dominio}
Ahora que ya tenemos avanzado en el desarrollo de nuestro sistema inteligente, podemos ver como ha ido aumentando el esquema del dominio, tanto en entidades como relaciones, que ahora contienen las relaciones lógicas con el conocimiento del dominio. En la Figura \ref{fig:DominioCompleto} muestra el esquema de dominio final.

\begin{figure}[H]
  \centering
  \includegraphics[scale=0.25,angle=90]{imaxes/Diagrama_Dominio_Completo.png}
  \caption{\label{fig:DominioCompleto}Esquema completo del dominio}
\end{figure}
