%%%%%%%%%%%%%%%%%%%%%%%%%%%%%%%%%%%%%%%%%%%%%%%%%%%%%%%%%%%%%%%%%%%%%%%%
% Plantilla TFG/TFM
% Universidad de A Coruña. Facultad de Informática
% Realizado por: Welton Vieira dos Santos
% Modificado: Welton Vieira dos Santos
% Contacto: welton.dossantos@udc.es
%%%%%%%%%%%%%%%%%%%%%%%%%%%%%%%%%%%%%%%%%%%%%%%%%%%%%%%%%%%%%%%%%%%%%%%%


\chapter{Modelo de conocimiento}
\newpage
\section{Fase de identificación}
\subsection{Tareas del formulario OM-3}
Las tareas elegidas para este modelo conceptual han sido las tareas 2, 3 y 4 del OM-3 (Tabla \ref{tab:IdentificacionOM3}), que corresponden con \textbf{Generar lista de entregas según ruta asignada}, \textbf{Determinar los recursos disponibles por la sucursal MRW para entregar según ruta} y \textbf{Revisión y validación de la distribución de la paquetería}.

\begin{table}[H]
  \centering
  \resizebox{15,0cm}{!}{
    \begin{tabular}{|c|c|c|c|c|c|c|}
      \hline
      \multicolumn{3}{|c}{\textbf{Modelo de Organización}} & \multicolumn{4}{|c|}{\textbf{Formulario OM-3: Descomposición de los Procesos}}\\
      \hline \hline
      \textsc{N\textordmasculine} & \textsc{Tarea} & \textsc{Realiza\-da por} & \textsc{¿Dónde?} & \textsc{Recursos de Conocimiento} & \textsc {¿In\-ten\-si\-va en Conocimiento?} & \textsc{Im\-por\-tan\-cia} \\
      \hline
            
      2 & \multicolumn{1}{|p{4.0cm}|}{\centering Generar lista de entregas según ruta asignada} & \multicolumn{1}{|p{3.0cm}|}{\centering Repartidor} &  \multicolumn{1}{|p{4.0cm}|}{\centering En PC del inversor (usuario)} & \multicolumn{1}{|p{5.0cm}|}{\centering Experiencia en reparto de paquetes según la ruta asignada} & Sí (bajo) & Máxima \\
      \hline
      3 & \multicolumn{1}{|p{4.0cm}|}{\centering Determinar los recursos disponibles por la sucursal MRW para entregar según ruta} & \multicolumn{1}{|p{3.0cm}|}{\centering Equipo Directivo, Repartidor experimentado} &  \multicolumn{1}{|p{4.0cm}|}{\centering En la nave de la sucursal de MRW} & \multicolumn{1}{|p{5.0cm}|}{\centering Experiencia en distribución de recursos. Utilizar teoria de programación dinámica} & Sí (elevado) & Máxima \\
      \hline
      4 & \multicolumn{1}{|p{4.0cm}|}{\centering Revisión y validación de la distribución de la paquetería} & \multicolumn{1}{|p{3.0cm}|}{\centering Equipo Directivo} &  \multicolumn{1}{|p{4.0cm}|}{\centering En la nave de la sucursal de MRW} & \multicolumn{1}{|p{5.0cm}|}{\centering Experiencia en distribución de recursos. Utilizar teoria de programación dinámica} & Sí (moderado) & Máxima \\
      \hline
    \end{tabular}
  }
	\caption{\label{tab:IdentificacionOM3}Tareas elegida para el modelo de conocimiento}
\end{table}

\subsection{Glosario de términos}
\begin{itemize}
	\item \textbf{Paquete:} Envoltorio que contiene cualquier tipo de objeto. Se utiliza tambien para indicar los sobres.
	\item \textbf{Ruta:}
	\item \textbf{Destinatario:}
	\item \textbf{Remitente:}
	\item \textbf{Incidencia:} Impendimento de entrega.
	\item \textbf{Plataforma:} Local de asignación de envíos.
	\item \textbf{Envios:} Utilizado para agrupar uno o mas paquetes a un destinatario.
	\item \textbf{Recogidas:} Utilizado para recoger paquete de un remitente.
\end{itemize}
